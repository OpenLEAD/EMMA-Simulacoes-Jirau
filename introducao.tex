\section{Introdução}
A análise de viabilidade técnica do projeto EMMA necessita da construção do
ambiente de simulação da área da turbina em que a operação de revestimento será
realizada. Além disso, há a necessidade da modelagem do manipulador, bases e
trilhos para a construção da solução, neste ambiente. Por último, deve ser
realizada análises de espaço de trabalho, cinemática e dinâmica do manipulador,
verificando possíveis colisões, e restrições de espaço na área de trabalho
devido ao controle (manipulabilidade) do robô. 

O resultado da simulação revela as áreas da pá que podem ser revestidas, as
áreas de mais difícil acesso, e as posições da base do manipulador para a
execução bem sucedida da operação.
