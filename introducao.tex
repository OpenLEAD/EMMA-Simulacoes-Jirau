\section{Introdução}
A análise de viabilidade técnica do projeto EMMA necessita da construção do
ambiente de simulação da área da turbina em que a operação de revestimento será
realizada. Além disso, há a necessidade da modelagem do manipulador, bases e
trilhos para a construção da solução, neste ambiente. Por último, deve ser
realizada análises de espaço de trabalho, cinemática e dinâmica do manipulador,
verificando possíveis colisões, e restrições de espaço na área de trabalho
devido ao controle (manipulabilidade) do robô. 

Nesta análise de viabilidade técnica, foi utilizada a 
multi-plataforma de arquitetura de software de código aberto Open
Robotics and Animation Virtual Environment (OpenRAVE). OpenRAVE é voltado para
aplicações de robôs autônomos e possui simulação 3D, visualização, planejamento,
e controle. A arquitetura de plugin permite ao usuário desenvolver controladores
customizados ou estender funcionalidades. O desenvolvedor pode se
concentrar em planejamento e aspectos específicos do problema, sem necessitar
gerenciar explicitamente os detalhes de cinemática e dinâmica do robô, e
detecção de colisão ou atualização do mundo. Além disso, OpenRAVE pode ser usado
em conjunção com populares pacotes de robótica, como ROS, Player e MatLab
\cite{diankov2008openrave}. 

O ambiente e componentes da turbina necessários para o ambiente de simulação
foram modelados no Software de projeto CAD 3D SolidWorks a partir dos desenhos
técnicos da Energia Sustentável do Brasil (ESBR). Foram modelados: aro câmara,
tubo de sucção, rotor, distribuidor, cone, pontos de acesso e pás da turbina.
Posteriormente, após visita à unidade geradora, verificou-se que o modelo da pá
continha inconsistências com a pá real, portanto foi realizado um mapeamento 3D
do ambiente e da pá pelo sensor 3D Laser Scanner FARO Focus3D. Todo o ambiente é
importado pelo ambiente de simulação OpenRAVE.

O resultado da simulação revela as áreas da pá que podem ser revestidas, as
áreas de mais difícil acesso, e as posições da base do manipulador para a
execução bem sucedida da operação.